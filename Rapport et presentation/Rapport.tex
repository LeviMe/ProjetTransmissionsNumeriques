\documentclass[a4paper,11pt]{article}
\usepackage[utf8]{inputenc}

\usepackage{amsmath,amssymb}
\usepackage{amsthm,mathrsfs}
\usepackage{tcolorbox}
\usepackage{url}
\usepackage[left=2cm,right=2cm,top=2cm,bottom=2cm]{geometry}
\usepackage[francais]{babel}
\newcommand{\ds}{\displaystyle}
\title{Rapport projet Transmissions numériques}
\author{MEGUIRA Levi \\
	CLAYBROUGH Jonathan}

\usepackage{pgfplots,tcolorbox,listings}
\tcbuselibrary{fitting}
\tcbuselibrary{skins}
\tcbuselibrary{listings}

\usepackage{color} %red, green, blue, yellow, cyan, magenta, black, white
\definecolor{mygreen}{RGB}{28,172,0} % color values Red, Green, Blue
\definecolor{mylilas}{RGB}{170,55,241}

\lstset{language=Matlab,%
    %basicstyle=\color{red},
    breaklines=true,%
    morekeywords={matlab2tikz},
    keywordstyle=\color{blue},%
    morekeywords=[2]{1}, keywordstyle=[2]{\color{black}},
    identifierstyle=\color{black},%
    stringstyle=\color{mylilas},
    commentstyle=\color{mygreen},%
    showstringspaces=false,%without this there will be a symbol in the places where there is a space
   % numbers=left,%
   basicstyle=\ttfamily,
    numberstyle={\tiny \color{black}},% size of the numbers
    numbersep=9pt, % this defines how far the numbers are from the text
    emph=[1]{for,end,break},emphstyle=[1]\color{red}, %some words to emphasise
    %emph=[2]{word1,word2}, emphstyle=[2]{style},
}
\newtcblisting{lstun}
{colback=blue!5!white,
coltitle=white,
colframe=blue!20!white,
fonttitle={\large\bfseries},
enhanced, sharp corners,
listing only,
listing options={language=Matlab,aboveskip=3mm,
belowskip=3mm,
basicstyle={\small\ttfamily},
keywordstyle={\ttfamily \color{blue} \bfseries},
stringstyle=\color{blue}}
}

\newcommand{\X}{\mathcal{X}}
\newcommand{\C}{\mathcal{C}}
\newcommand{\N}{\mathbb{N}}
\newcommand{\R}{\mathbb{R}}

\begin{document}
\maketitle

\begin{comment}
	\textbf{Remarque importante: Notre chaine de transmission ne \emph{donne pas l'air} de transmettre \emph{un} signal complexe mais bien deux signaux distincts qui subissent exactement les mêmes opérations. Afin de respecter au possible le schéma de l'énoncé ainsi que le principe de ne jamais répeter de code en Matlab, il importe de créer un signal numérique \emph{réelement complexe} (sans jeu de mot) à partir des bits de départ et de transmettre un seul et même objet d'une extrémité à l'autre. \\ Le caractère universel des fonctions Matlab suffit à faire appliquer toutes les fonctions utilisés dans ce code à un vecteur complexe définie par \texttt{symboles=bitsI+j*bitsQ}}
\end{comment}

\section*{Contexte}
Ce projet de communication numérique vise à simuler sous Matlab une chaîne de transmissions au standard DVB-S. Il se divise en trois grandes parties que sont:
\begin{enumerate}
	\item La création d'un couple modulateur/ démodulateur.
	\item Le codage canal avec code RS, code convolutif poinçonnage et entrelacement.
	\item La synchronisation avec une structure en boucle fermée pour corriger les erreurs de phases.
\end{enumerate}

\tableofcontents

\section{Organisation du travail}
Nous étions un groupe de deux, alors plutôt que de travailler deux fois les mêmes choses, nous nous somme répartis les tâches à faire puisqu'elles étaient relativement indépendantes, puis avons mis en commun pour former la chaine entière.
\newline
Cette répartition fut possible grâce à la structure modulaire du projet. Toutes les opérations ont été effectué via la platforme Github, avec un dépot visible à l'adresse suivante \url{https://github.com/LeviMe/ProjetTransmissionsNumeriques}.
\section{Structure du code}

\section{Chaine de base}

\section{Modulation-démodulation}
L'implémentation de cette première partie se fait avec une chaine passe-bas équivalente à la chaine de transmission sur porteuse du cas simulé. Elle envoie un signal complexe plutôt que deux signaux réels en quadrature, et ce de la façon suivante:
\begin{enumerate}
	\item Des bits d'informations sont modulés en QPSK, c'est à dire sous la forme de symboles complexes $c_k=a_k+jb_k$ ou $a_k,b_k \in \{-1,+1\}^2$
	\item Ces symboles sont passés par un filtre de mise en forme en racine de cosinus surélevé et envoyés sur un canal.
	\item Le canal ajoute un bruit gaussien d'une densité spectrale de puissance spécifiée.
	\item Le signal atteint le récepteur ou il est passé par un filtre adapté et échantionné aux instants optimaux.
	\item Enfin il est divisé en ses parties réelles et imaginaires puis passé par un détecteur à seuil pour prendre les décisions et retrouver les symboles QPSK d'emissions.
\end{enumerate}

\section{Bruit canal}

\section{Codage canal}
Le codage canal se décompose en plusieurs parties que sont dans l'ordre:
\begin{enumerate}
	\item Un code de Reed-Solomon réduit (204,188).
	\item Un entrelaceur conforme au standard DVB-S c'est à dire de \emph{paramètres} $17$ et $12$.
	\item Un code convolutif de polynômes générateur $171, 133$, qui est poinçonné via la matrice $[1,1,0,1]$.
\end{enumerate}

Toutes ces fonctions sont implémentés dans une seule est même fonction \texttt{codage} qui prend un vecteur de \texttt{bits} et tous les interrupteurs relatifs aux étapes de son fonctionnement et renvoie le vecteur codé. \\
Le fonctionnement par étapes consiste à initialiser un \texttt{bits\_codes}
à \texttt{bits} et à lui faire subir toutes les opérations de codages suivant la configuration d'interrupteurs choisie. C'est lui qui sera renvoyé en sortie de la fonction.

\begin{lstun}
function [bits_codes] = codage(bits, RS_encoding, interleaving, puncturing, convencoding)
\end{lstun}
\subsection{Code de Reed Solomon}
Le code de Reed-Solomon est par essence non-binaire; il prend en entrée des symboles qui dans notre cas seront des octets. Pour effectuer la conversion à partir de valeurs binaires, on va utiliser les fonctions \texttt{reshape} puis \texttt{bi2de} et obtenir ainsi le vecteur \texttt{symboles} huit fois plus cours. \\
Ce vecteur sera encodé pour donner un vecteur \texttt{symboles\_codes} puis à son tour converti en valeurs binaires via une opération analogue à la première.

\begin{lstun}
	%comment
	if RS_encoding
		symboles=bi2de(reshape(bits,[],8));
		symboles_codes = step(enc,symboles);
		bits_codes = reshape(de2bi(symboles_codes),[],1);
	end
\end{lstun}

\subsection{Entrelacement}
L'entrelacement au standard DVB-S fonctionne avec deux paramètres appelés \texttt{nrows} et \texttt{slope} qui signifient ------------ et dont la valeur est fixée respectivement à 17 et 12. \\ De ces paramètres dépens la valeur du padding à introduire pour obtenir une transmission sans erreur notée \texttt{D}.

\begin{lstun}
	nrows = 17; slope = 12;
	D = nrows*(nrows-1)*slope;
	...
	if interleaving
	bits_codes = convintrlv([bits_codes;zeros(D,1)], nrows,slope);
	end
\end{lstun}

\subsection{Code convolutif et poinçonnage}
Le code convolutif requis par l'énoncé de paramètres $(7,1/2)$ et de polynômes générateurs $g_1=171_{oct}$ et  $g_2=133_{oct}$. \\
La syntaxe de son implémentation Matlab se fait via la création d'un objet d'encodage \texttt{trellis} qui peut accueillir ou non une matrice de poinçonnage. C'est ce qui est fait dans le code suivant où les deux configurations sont envisagées. \\
On aurait également pu procéder en l'absence de poinçonnage avec une matrice $[1,1,1,1]$, c'est à dire celle qui conserve le signal de départ.

\begin{lstun}
puncturing_matrix = [1 1 0 1];
trellis = poly2trellis([7],[171 133]);
-------
if puncturing
	bits_codes = convenc(bits_codes,trellis,puncturing_matrix);
else
if convencoding
	bits_codes = convenc(bits_codes,trellis);
end
end
\end{lstun}

\subsection{Décodage}
Le décodage se fait via une fonction éponyme qui effectue toutes les opérations duales de la chaine de codage dans l'ordre inverse. \\
Son implémentation a été relativement simple compte-tenu du fait que Matlab fournis simultanément les couples de fonctions qui possèdent ainsi une syntaxe identique.

\section{Retard de phase}
Le retard de phase est relativement trivial à implémenter, puisqu'il suffit de prendre le signal complexe et de le multiplier par $e^{j \phi}$. 

\begin{lstun}
	function [ symb_d ] = Dephasage( phi, ecartFreq, symb )
	d_phi = phi.*ones(1,length(symb)) + [1:length(symb)].*2.*pi.*ecartFreq;
	symb_d = symb.*exp(j*d_phi);
	end
\end{lstun}

\section{Correction du retard de phase}

\begin{lstun}
	function [ symb_corrige ] = PLL(A, B, symb)
	%PLL Detection et correction du dephasage
	%   Detection et correction du dephasage du signal du a la difference de
	%   frequence des horloges a l'emission et la reception
	
	symb_corrige = zeros(1, length(symb));
	phi_est = zeros(1, length(symb)+1);
	out_det = zeros(1, length(symb));
	w = zeros(1, length(symb));
	NCO_mem=0;      % initialisation NCO
	filtre_mem=0;   % initialisation de la memoire du filtre
	
	%PLL
	for i=1:length(symb)
	symb_corrige(i) = symb(i)*exp(-j*phi_est(i));
	out_det(i)= -imag(symb_corrige(i).^4);
	
	% filtre de boucle
	w(i)=filtre_mem+out_det(i); % memoire filtre + sortie detecteur 
	filtre_mem=w(i);            
	out_filtre=A*out_det(i)+B*w(i);   % sortie du filtre a l'instant i :  F(z)=A+B/(1-z^-1)
	
	%NCO
	phi_est(i+1)=(out_filtre+NCO_mem); % N(z)=1/(z-1) 
	NCO_mem=phi_est(i+1);
	end
	end	
\end{lstun}

\section{Quelques résultats}
Ayant implémenté la chaine de façon configurable, nous avons pu faire varier les paramètres de tests pour sortir différents graphiques. 
On fait d'abord un parcours intégral 

\section{Conclusion}


\newpage

\begin{verbatim}

%raw Data
taux d erreurs sans codage=

    0.0546    0.0504    0.0429    0.0368    0.0322

taux d erreurs avec codage RS et interleaving =

    0.0548    0.0542    0.0436    0.0401    0.0288

taux d erreurs avec codage RS sans interleaving=

    0.0571    0.0556    0.0442    0.0395    0.0319
taux d erreurs scode =

    0.0564    0.0538    0.0427    0.0375    0.0274

taux d erreurs avec RS, interleaving, convencoding  =

    0.0529    0.0426    0.0370    0.0196    0.0147

taux d erreurs =

    0.2094    0.1934    0.1561    0.1106    0.0649

\end{verbatim}


\begin{center}
\begin{tikzpicture}[baseline]
\begin{axis}[width=8.5cm,
title=un titre à modifier,
xlabel={$f$},
ylabel={$g$}]
%\addplot table[mark=none,x expr =\coordindex+1, y index=0] {fichier.txt};
%cas d'un tableau à une entrée, l'indice (#de ligne) de la valeur étant son abcisse
%\addplot table[mark=none,x=abs, y=ord] {fichier.txt};
%cas d'un tableau à plus de deux entrées, avec tracées des colonnes etiquetées à la première ligne respectivement par abs et ord.
\addplot table[mark=none] {fichier.txt};
%cas d'un tableau à deux entrées.
\end{axis}
\end{tikzpicture}
\end{center}
\section{conclusion}
\end{document}
